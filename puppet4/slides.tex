\titleSlide
%\frame[b]{%
%    \largepic{drawing.png}
%}
\frame{%
    \frametitle{\$::user}
    \framesubtitle{Julien Pivotto}
    \begin{itemize}
        \item{Open-Source consultant at {\inuits{}inuits\small.eu}}
        \item{Puppet User since 2011}
        \item{Speaker/attendee at Puppetcamps}
        \item{Member of the Belgian PUG}
        \item{Puppet core contributor}
        \item{Puppet(labs) modules contributor}
        \olditem{\textit{\ctr{@roidelapluie}} \ctr{on irc/twitter/github}}
    \end{itemize}

}
\inuitsSlide
%\CenterSlide{A short introduction to Puppet}
%
%\begin{iframe}[What is Puppet]
%\item A Configuration Management tool
%\item Server/Agent model or standalone
%\item Goal: Define the state of the infrastructure
%\item Automate common tasks and deployments
%\end{iframe}
%
%\begin{iframe}[Terminology]
%\item Resource: The definition of an "item" (package, service, file\dots)
%\item Manifests: A file that contains a list of resources
%\item Module: A collection of manifests (e.g a module can manage an apache setup)
%\item Catalog: A list of resources that a node has to apply, that describes the wanted state
%\end{iframe}
%\begin{iframe}[In practice]
%\item A client requests a catalog from the server
%\item The server sends a catalog for that node with all the resources
%\item The agent applies the catalog: install packages, files, starts services,\dots
%\end{iframe}
%\begin{iframe}[Other features]
%\item PuppetDB: a storage for reports, resources\dots
%\item Mcollective: an orchestration framework
%\item Facter: collects informations about the current system and expose them with an api
%\item Hiera: split data and code
%\end{iframe}
%
\CenterSlide{Introduction}


\begin{iframe}[Puppetmaster since 2011]
\item Puppet 2.6/2.7
\item Reports stored with activerecord
\item Puppet Dashboard
\item No hiera
\item No PuppetDB
\end{iframe}
\begin{iframe}[Puppet 2.7 - June 2011]
\item Ruby 1.9 support
\item Apache 2.0 Licence
\item The end of "random" catalog order
\item Support for network devices
\item Deprecation of dynamic variables
\end{iframe}
\begin{iframe}[Puppet 3.0.0 - October 2012]
\item Support for 3rd parties gems
\item Native hiera support
\item Automatic hiera lookups
\item Remove dynamic scope lookup
\item Semantic versioning
\end{iframe}
\begin{iframe}[Puppet 3.x series]
\item Future parser (3.2.0)
\item "contain" keyword (3.4.0)
\item directory environments (3.5.0)
\end{iframe}
\frame[b]{%
    \largepic{4421810399_75d27102b4_o.jpg}
\begin{center}\Huge{}\ctr{Puppet 4}\end{center}
\sethlcolor{black}
\flushright\minuscule{Licensed under a Creative Commons Attribution 2.0 License\\https://www.flickr.com/photos/mujitra/4421810399}
}
\begin{iframe}[The big picture]
\item New DSL language
\item All-in-one Package
\item Puppetserver
\end{iframe}
\frame[b]{%
    \largepic{code.png}
\begin{center}\Huge{}\ctr{A new DSL}\end{center}
\sethlcolor{black}
\flushright\minuscule{Apache 2.0 License\\https://github.com/ke4qqq/puppet-cloudstack}
}

\begin{frame}\frametitle{Domain Specific Language}
    \begin{center}
        \huge{The goal of the Puppet DSL stays the same: describe the final state of your infrastructure.}
    \end{center}
\end{frame}

\begin{frame}
    \begin{center}
        \huge{The new DSL gets closer to other programming languages: types, iterations, predictive ordering\dots}
    \end{center}
\end{frame}
\frame[b]{%
    \largepic{19543885696_aa036a416e_k.jpg}
\begin{center}\Huge{}\ctr{Ordering}\end{center}
\sethlcolor{black}
\flushright\minuscule{Licensed under a Creative Commons Attribution-ShareAlike 2.0 License\\https://www.flickr.com/photos/lambertclement/19543885696}
}

\begin{frame}
    \frametitle{order.pp}
    \lstset{language=Puppet}
    \lstset{%
         basicstyle=\LARGE,
     }
 \lstinputlisting{ex1/test.pp}
 \end{frame}
\begin{frame}
    \frametitle{Puppet 3.x}
 \lstinputlisting{ex1/38.txt}
 \end{frame}
\begin{frame}
    \frametitle{Puppet 4.x}
 \lstinputlisting{ex1/42.txt}
 \end{frame}
 \begin{iframe}[Ordering]
\item Always the same order
\item Puppet 3.x: based on title-hash
\item Puppet 4.x: order seen in the manifest
\end{iframe}
 \begin{iframe}[Puppet 4 ordering]
 \item Time saver
\item "Hides" complexity
\item Align with other languages
\item Still supports Require, Notify, \dots
\item "real" dependencies should still be explicitely set (e.g cron job only installed if packages is installed)
\end{iframe}

% types

\frame[b]{%
    \largepic{6338281766_68b200c61b_b.jpg}
\begin{center}\Huge{}\ctr{(Native) Data Types}\end{center}
\sethlcolor{black}
\flushright\minuscule{Licensed under a Creative Commons Attribution 2.0 License\\https://www.flickr.com/photos/ionics/6338281766}
}
\begin{iframe}[Data types are not new]
\item Previously included in stdlib, 3rd party module
\item stdlib contains functions to convert data types, check data types
\item Nothing in Puppet Core
\item Difficult to check the type of all the parameters
\end{iframe}
 \begin{frame}
     \frametitle{Data Types}
     \begin{center}
         \LARGE{\texttt{String}}\pause\\
         \LARGE{\texttt{Numeric}: \texttt{Integer}, \texttt{Float}}\pause\\
         \LARGE{\texttt{Boolean}}\pause\\
         \LARGE{\texttt{Array}}\pause\\
         \LARGE{\texttt{Hash}}\pause\\
         \LARGE{\texttt{Regexp}}\pause\\
         \LARGE{\texttt{Undef}}\pause\\
         \LARGE{\texttt{Catalogentry}: \texttt{Class}, \texttt{Resource}}
    \end{center}
\end{frame}

\begin{frame}\frametitle{String}
\begin{table}
\begin{tabular}{rl}
\texttt{String} & Any string \\\pause
\texttt{String[2]} & Any string with at least 2 characters \\\pause
\texttt{String[0,2]} & A string with at most 2 characters \\\pause
\texttt{String[4,6]} & A string with at least 4 and at most 6 characters \\\pause
\texttt{String[5,5]} & A string with 5 characters
\end{tabular}
\end{table}
\center{Convert to string with: \texttt{"\$\{myvariable\}"}}
\end{frame}
\begin{frame}\frametitle{Numeric}
    \begin{table}
        \begin{tabular}{rl}
\texttt{Numeric} & Float or integer \\\pause
\texttt{Numeric[-1,1]} & Float or Integer between -1 and 1 (incl) \\\pause
\texttt{Integer} & Any integer (from $-\infty$ to $\infty$) \\\pause
\texttt{Integer[0]} & Positive Integer \\\pause
\texttt{Integer[default, 0]} & Negative Integer \\\pause
\texttt{Integer[-1,1]} & Integer between -1 and 1 (incl)  \\\pause
\texttt{Float} & Floating point number \\\pause
\texttt{Float[-3,3.14]} & Float between -3 and 3.14 (incl) \\
\end{tabular}
    \end{table}
\center{Convert to Integer with the \texttt{scanf} function or \texttt{0+\$variable}}
\end{frame}
\begin{frame}\frametitle{Hash}
    \begin{table}
        \begin{tabular}{c}
\texttt{Hash} \\ A Hash (key, value pairs)
\vspace{4mm} \\\pause
            \texttt{Hash[String, String]} \\ Hash with String as keys and values
\vspace{4mm} \\\pause
            \texttt{Hash[String, Integer]} \\ String as keys, Integer as values
\vspace{4mm} \\\pause
            \texttt{Hash[String, Integer, 1]} \\ Non empty hash
\vspace{4mm} \\\pause
            \texttt{Hash[String, Integer, default, 5]} \\ Hash with at most 5 entries
\end{tabular}
    \end{table}
\end{frame}
\begin{frame}\frametitle{Array}
    \begin{table}
        \begin{tabular}{c}
            \texttt{Array} \\ An array (list)
\vspace{4mm} \\\pause
            \texttt{Array[String]} \\ Array with String
\vspace{4mm} \\\pause
            \texttt{Array[String, 1]} \\ Non empty Array
\vspace{4mm} \\\pause
            \texttt{Array[String, default, 5]} \\ Array with at most 5 entries
\end{tabular}
    \end{table}
\end{frame}
\begin{frame}\frametitle{Undef}
    \begin{center}
        \LARGE
        Eventually! A clear type for Undef (was very confusing in the past: nil? string? symbol?)
    \end{center}
\end{frame}
\frame[b]{%
    \largepic{4666231979_2644531af8_b.jpg}
\begin{center}\Huge{}\ctr{Abstract Data Types}\end{center}
\sethlcolor{black}
\flushright\minuscule{Licensed under a Creative Commons Attribution 2.0 License\\https://www.flickr.com/photos/webtreatsetc/4666231979}
}
\begin{frame}\frametitle{Optional}
    \begin{center}
\LARGE
\texttt{Optional[String]}: String or Undef\\
\texttt{Optional[Integer]}: Integer or Undef
    \end{center}
\end{frame}
\begin{frame}\frametitle{NotUndef}
    \begin{center}
\LARGE
Everything but Undef.\\
\texttt{NotUndef[Data]}: Anything that matches Data but not Undef.\\
Example: \texttt{[Undef]}
    \end{center}
\end{frame}
%\texttt{NotUndef}
%\texttt{Variant[Integer, String]}: Integer or String
%\item {\texttt{Pattern[/A-Z/, \\d]}: String that matches regex}
%\item {\texttt{Enum["foo", "bar"]}: String "foo" or "bar"}
%\item {\texttt{Tuple[String, Integer]}: An array (["a", 1])}
%\item {\texttt{Struct}: Advanced hashes}
\begin{frame}\frametitle{Variant}
    \begin{center}
\LARGE
\texttt{Variant[Integer, String]: An Integer or a String}
    \end{center}
\end{frame}
\begin{frame}\frametitle{Pattern}
    \begin{center}
\LARGE
\texttt{Pattern[Regex, Regex]}: Strings match regexes
    \end{center}
\end{frame}
\begin{frame}\frametitle{Enum}
    \begin{center}
\LARGE
\texttt{Enum["true", "false"]: "true" or "false" (strings)}
    \end{center}
\end{frame}
\begin{frame}\frametitle{Tuple}
    \begin{center}
\LARGE
\texttt{Tuple[Integer, String]: An array like [1, "a"]}
    \end{center}
\end{frame}
\begin{frame}\frametitle{Struct}
    \begin{center}
\LARGE
Advanced hash.\\
\texttt{Struct[\{"username" => String, "uid" => Integer[0]\}]}\\
Matches \texttt{\{"username" => "tux", "uid" => 1\}}
    \end{center}
\end{frame}
%\item {\texttt{Scalar}: Number, String, Boolean, Regex}
%\end{iframe}
%\begin{iframe}[Parent Data Types]
%\item {\texttt{Data}: Scalar + Undef, Array[Data], Hash[Scalar, Data]}
%\item {\texttt{Collection}: Hash, Array}
%\item {\texttt{Catalogentry}: Resource, Class}
%\item {\texttt{Any}}
%\end{iframe}
\begin{frame}\frametitle{Scalar}
    \begin{center}
\LARGE
Same as \texttt{Variant[Numeric, String, Boolean, Regex]}
    \end{center}
\end{frame}
\begin{frame}\frametitle{Data}
    \begin{center}
\LARGE
Same as \texttt{Variant[Scalar, Undef, Array[Data], Hash[Scalar, Variant[Data]]}
    \end{center}
\end{frame}
\begin{frame}\frametitle{Any}
    \begin{center}
\LARGE
Anything
    \end{center}
\end{frame}
\frame[b]{%
    \largepic{7439512656_ae58012e17_k.jpg}
\begin{center}\Huge{}\ctr{Playing with types}\end{center}
\sethlcolor{black}
\flushright\minuscule{Licensed under a Creative Commons Attribution 2.0 License\\https://www.flickr.com/photos/jdhancock/7439512656}
}
\begin{frame}
    \frametitle{Types in conditions}
    \lstset{language=Puppet}
    \lstset{%
         basicstyle=\large,
     }
 \lstinputlisting{types/types.pp}
 \end{frame}
\begin{frame}
    \frametitle{Types in case statements}
    \lstset{language=Puppet}
    \lstset{%
         basicstyle=\large,
     }
 \lstinputlisting{types/types2.pp}
 \end{frame}
\begin{frame}
    \frametitle{Types in parameters}
    \lstset{language=Puppet}
    \lstset{%
         basicstyle=\large,
     }
 \lstinputlisting{types/types3.pp}
 \end{frame}
\begin{frame}
    \frametitle{Type checking}
    \lstset{language=Puppet}
    \lstset{%
         basicstyle=\large,
     }
 \lstinputlisting{types/types4.pp}
 \end{frame}

% merge
\frame[b]{%
    \largepic{2295022876_3cb19f4fb7_o.jpg}
\begin{center}\Huge{}\ctr{Array, Hash merge}\end{center}
\sethlcolor{black}
\flushright\minuscule{Licensed under a Creative Commons Attribution 2.0 License\\https://www.flickr.com/photos/willmx/2295022876}
}
\begin{frame}
    \frametitle{Hashes}
    \lstset{language=Puppet}
    \lstset{%
         basicstyle=\large,
     }
 \lstinputlisting{ex2/1.pp}
 \end{frame}
\begin{frame}
    \frametitle{Arrays}
    \lstset{language=Puppet}
    \lstset{%
         basicstyle=\large,
     }
 \lstinputlisting{ex2/2.pp}
 \end{frame}

 \begin{iframe}[Merges]
 \item No more need for stdlib
\item Cleaner code
\end{iframe}


\frame[b]{%
    \largepic{3269784239_e208c5b968_o.jpg}
\begin{center}\Huge{}\ctr{Functions}\end{center}
    \flushright\minuscule{\color{black}Licensed under a Creative Commons Attribution 2.0 License\\https://www.flickr.com/photos/zzpza/3269784239}
}
\begin{frame}
    \frametitle{Puppet 4 Function}
    \lstset{language=ruby}
    \lstset{%
         basicstyle=\large,
     }
 \lstinputlisting{fc/fc.rb}
 \end{frame}
\begin{frame}
    \frametitle{Puppet 4 Function}
    \lstset{language=ruby}
    \lstset{%
         basicstyle=\tiny,
     }
 \lstinputlisting{fc/fc2.rb}
 \end{frame}
\frame[b]{%
    \largepic{8114399667_30e7770e7d_k.jpg}
\begin{center}\Huge{}\ctr{Loops and new built-in functions}\end{center}
    \flushright\minuscule{Licensed under a Creative Commons Attribution-ShareAlike 2.0 License\\https://www.flickr.com/photos/squeaks2569/8114399667}
}

\begin{frame}
    \frametitle{Loops: Array}
    \lstset{language=Puppet}
    \lstset{%
         basicstyle=\large,
     }
 \lstinputlisting{loops/1.pp}
 \end{frame}
\begin{frame}
    \frametitle{Loops: Hash}
    \lstset{language=Puppet}
    \lstset{%
  %       basicstyle=\large,
     }
 \lstinputlisting{loops/2.pp}
 \end{frame}
\begin{frame}
    \frametitle{Loops: reduce}
    \lstset{language=Puppet}
    \lstset{%
 %        basicstyle=\large,
     }
 \lstinputlisting{loops/3.pp}
 \end{frame}
\begin{frame}
    \frametitle{with: "private" scope}
    \lstset{language=Puppet}
    \lstset{%
         basicstyle=\large,
     }
 \lstinputlisting{misc/1.pp}
 \end{frame}
\begin{frame}
    \frametitle{filter}
    \lstset{language=Puppet}
    \lstset{%
%         basicstyle=\large,
     }
 \lstinputlisting{misc/2.pp}
 \end{frame}
\begin{frame}
    \frametitle{access resource parameters}
    \lstset{language=Puppet}
    \lstset{%
%         basicstyle=\large,
     }
 \lstinputlisting{misc/3.pp}
 \end{frame}
\begin{frame}
    \frametitle{ruby-style~.~syntax (chaining)}
    \lstset{language=Puppet}
    \lstset{%
%         basicstyle=\large,
     }
 \lstinputlisting{misc/4.pp}
 \end{frame}
\begin{frame}
    \frametitle{HEREDOC}
    \lstset{language=Puppet}
    \lstset{%
%         basicstyle=\large,
     }
 \lstinputlisting{misc/5.pp}
 \end{frame}
\frame[b]{%
    \largepic{5941302441_ce26de10a6_o.jpg}
\begin{center}\Huge{}\ctr{Hashes and Defaults}\end{center}
    \flushright\minuscule{Licensed under a Creative Commons Attribution-ShareAlike 2.0 License\\https://www.flickr.com/photos/mikecogh/5941302441}
}

\begin{frame}
    \frametitle{params as a hash}
    \lstset{language=Puppet}
    \lstset{%
%         basicstyle=\large,
     }
 \lstinputlisting{misc/6.pp}
 \end{frame}
\begin{frame}
    \frametitle{defaults}
    \lstset{language=Puppet}
    \lstset{%
%         basicstyle=\large,
     }
 \lstinputlisting{misc/7.pp}
 \end{frame}
\begin{frame}
    \frametitle{Params + defaults as hash}
    \lstset{language=Puppet}
    \lstset{%
%         basicstyle=\large,
     }
 \lstinputlisting{misc/8.pp}
 \end{frame}
\frame[b]{%
    \largepic{3267227227_660b6ab4f4_b.jpg}
\begin{center}\Huge{}\ctr{Templates}\end{center}
    \flushright\minuscule{Licensed under a Creative Commons Attribution 2.0 License\\https://www.flickr.com/photos/mrbill/3267227227}
}
 \begin{iframe}[Templates]
 \item Can be written in Puppet Code (EPP)
\item Gets their own variables
\item Only useful since the new DSL
\item function \texttt{epp} and \texttt{inline\_epp}
\end{iframe}
\begin{frame}
    \frametitle{EPP templates}
    \lstset{language=Puppet}
    \lstset{%
         basicstyle=\large,
     }
 \lstinputlisting{misc/tmp.pp}
 \end{frame}
\begin{frame}
    \frametitle{EPP templates}
    \lstset{%
         basicstyle=\large,
     }
 \lstinputlisting{misc/tmp.epp}
 \end{frame}

\frame[b]{%
    \largepic{5299374697_6b97479c95_b.jpg}
\begin{center}\Huge{}\ctr{All-in-One package}\end{center}
    \flushright\minuscule{Licensed under a Creative Commons Attribution 2.0 License\\https://www.flickr.com/photos/creative\_tools/5299374697}
}

\begin{iframe}[All-in-One package: Facts]
\item All the client is in one RPM
\item Repository is called PC1
\item Puppet Collection 1
\item puppet-agent RPM contains ruby and all the deps
\item Everything is under /etc/puppetlabs, /opt/puppetlabs, \dots
\end{iframe}
\begin{iframe}[All-in-One package: Pros]
\item Everyone gets the same ruby version
\item One RPM, one repository
\item Everything can be optimized
\item Only one version to check for consistency
\end{iframe}
\begin{iframe}[All-in-One package: Cons (1/2)]
\item Not rebuildable (toolchain (ezbake/vanagon) not released/not open-source)
\item PL can patch Ruby, preventing people that use other methods to get the same behaviour
\item We are dependent on PL for updates (e.g OpenSSL)
\item Hidden version numbers (Puppet-agent version is semantic but abstract)
\item Not reusing shared libs
\end{iframe}
\begin{iframe}[All-in-One package: Cons (2/2)]
\item How can we guess such paths? \texttt{/opt/puppetlabs/puppet/cache} should have been at least \texttt{/var/opt/puppetlabs/puppet}
\end{iframe}
\begin{iframe}[Batteries included]
\item Hiera, Augeas
\item Facter, CFacter
\item Mcollective
\item OpenSSL
\item Ruby and gem dependencies
\end{iframe}
\begin{iframe}[The case of Fedora]
\item Fedora 23 ships Puppet 4
\item NOT using the AIO package
\item Might be rebuilt one day for EL
\end{iframe}
% AIO

\frame[b]{%
    \largepic{6362329323_56c96f68c1_o.jpg}
\begin{center}\Huge{}\ctr{Migration and stuff}\end{center}
    \flushright\minuscule{Licensed under a Creative Commons Attribution-ShareAlike 2.0 License\\https://www.flickr.com/photos/photos/26116471@N03/6362329323}
}

\begin{iframe}[Puppetserver]
\item A Puppet master written in clojure
\item Runs on top of a jvm
\item Put it behind a reverse proxy
\item Slow start but better performances
\item Not much to say, it stays out of my way so it's good
\item Puppetserver 2.1 is compatible with Puppet 3 and 4 clients
\end{iframe}
\begin{iframe}[Prepare yourself]
\item Update to latest Puppet 3.x
\item Use the Future parser
\item Update to latest Puppetserver
\end{iframe}
\begin{iframe}[Update your clients]
\item https://github.com/puppetlabs/puppetlabs-puppet\_agent
\item Do not forget to edit your cronjobs (change PATH to /usr/bin:/opt/puppetlabs/bin)
\item puppet will be in /opt/puppetlabs/bin
\end{iframe}
\begin{iframe}[Everything is not there yet]
\item Puppetboard and Foreman does not work yet (new PuppetDB API)
\item Good news: Puppetexplorer does (I have not tested)
\end{iframe}

\frame[b]{%
    \largepic{19824185788_f8d7ae406a_k.jpg}
\begin{center}\Huge{}\ctr{Conclusion}\end{center}
    \flushright\minuscule{Licensed under a Creative Commons Attribution-ShareAlike 2.0 License\\https://www.flickr.com/photos/olivierpechenet/19824185788}
}
\begin{iframe}[Good job :-)]
\item Good job, Puppetlabs (and community)
\item The new DSL was a hard work
\item It works well
\item I have seen only minor bugs
\item It is still a lot backward-compatible (with clean code\dots)
\item Please open-source the toolchain
\end{iframe}

\begin{iframe}[Regarding the new DSL]
\item A lot of new powers
\item It's a language
\item Time to write new recommendations, best practices
\item It can look awesome or horrible
\item Puppet lint and code review to the rescue!
\end{iframe}

\begin{iframe}[Is it time to move?]
\item If you have a test environment, yes
\item Otherwise I would still wait for more bugfixes
\item But we are already at 4.2.1
\item You should a least prepare yourself
\item I recommend github.com/theforeman/puppet-puppet to manage puppet
\end{iframe}

\begin{iframe}[The new DSL]
\item Will not be everywhere
\item A lot of stuff could be done with stdlib
\item First in roles, profiles, node manifests
\item Waiting for PL to decomission 3.x for the rest
\item Maybe we will see puppet4-only modules (but is that what we want?)
\end{iframe}

\begin{iframe}[Puppet 4]
\item Full of new features
\item Now has a powerful DSL
\item Stays declarative
\item Is more predictive
\end{iframe}

\CenterSlide{Any Question?}
% Puppet server / rproxy / puppet-puppet

 % upgrade
\begin{iframe}[Bio]
\item http://puppet-on-the-edge.blogspot.com/
\item https://www.devco.net/
\item http://docs.puppetlabs.com/
\end{iframe}

\contactSlide



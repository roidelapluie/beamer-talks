\titleSlide

\frame{%
    \frametitle{\$::user}
    \framesubtitle{Julien Pivotto}
    \begin{itemize}
        \item{Open-Source consultant at {\inuits{}inuits\small.eu}}
        \item{Puppet User since 2011}
        \item{Speaker/attendee at Puppetcamps}
        \item{Puppet core contributor}
        \item{Puppet(labs) modules contributor}
        \olditem{\textit{\ctr{@roidelapluie}} \ctr{on irc/twitter/github}}
    \end{itemize}

}

\inuitsSlide

\begin{iframe}[IT Nowadays (?)]
\item Virtualization
\item Containers
\item Cloud
\item Stateless software
\item Scalable daemons
\end{iframe}
\begin{frame}
    \frametitle{Distributing Software}
    \begin{itemize}
        \item {Plain FTP (+SCM)}
        \item {Tarball}
        \item {Self-extracting tarball}
        \item {curl|bash}
        \item{Containers? What's inside?}
        \item {Packages (.deb,~.rpm,\dots)}
    \end{itemize}
\end{frame}
\frame[b]{%
    \largepic{entreprise.jpg}
    \begin{center}\Huge{}\ctr{Real World}\end{center}
    \sethlcolor{black}
    \flushright\minuscule{Licensed under a Creative Commons Attribution 2.0 License\\https://www.flickr.com/photos/kwarz/13293732384/}
}


\begin{iframe}[Software in my world]
\item Complex SW
\item Self-hosted
\item Stateful
\item Scalable
\item Lots of reqs and deps
\end{iframe}

\begin{iframe}[Distribution]
\item Software (Source code or binaries)
\item User guide
\item Installation guide
\item People who install the software
\end{iframe}

\begin{iframe}[Challenges of SW distribution]
\item Artifacts
\item Configuration
\item Security
\item HW requirements
\item SW requirements
\item Upgrades, Maintenance
\item Monitoring
\end{iframe}


\section{Introduction}
\frame[b]{%
    \largepic{packaging.jpg}
\begin{center}\Huge{}\ctr{Artifacts}\end{center}
\sethlcolor{black}
\flushright\minuscule{Licensed under a Creative Commons Attribution-ShareAlike 2.0 License\\https://www.flickr.com/photos/halfbisqued/2353845688}
}
\begin{frame}
    \frametitle{Artifacts: Packaging}
    \begin{itemize}
        \item {Consistency checks, file lists}
        \item {Dependencies resolving}
        \item {Repositories}
        \item {GPG-Signing}
        \item {Lots of tools}
        \item {Versioning}
        \item {Unique artifacts, reproducible build}
        \item {CfgMgmt integration (Puppet, Chef\dots)}
    \end{itemize}
\end{frame}

\begin{iframe}[Dependencies]
\item{Packaged also}
\item{In versioned repositories}
\item{Test your dependencies}
\item{Distribution, Upstream packages}
\item{Mirror and cherry pick from upstream repos}
\item{Limit the number of dependencies}
\end{iframe}


\begin{iframe}[Automation]
\item Automate all the things
\item OS, Monitoring, Application
\item Reproducable builds
\item Repositories management
\end{iframe}

\begin{iframe}[Cultural changes]
\item No more manual work
\item Use an appropriate toolchain
\item Manual action = Error prone
\item What did I change 3 years ago?
\end{iframe}

\begin{iframe}[Documentation]
\item No more complex install guides
\item Puppet training
\item Valid for several products
\end{iframe}

\section{Scope}
\frame[b]{%
    \largepic{puppet.jpg}
    %\sethlcolor{black}
    \flushright\minuscule{Licensed under a Creative Commons Attribution 2.0 License\\https://www.flickr.com/photos/jimmcd/4859841581}
}

\begin{iframe}[Puppet]
\item Widely used Automation tool\pause
\item Very mature\pause
\item Client/Server mode\pause
\item Standalone mode\pause
\item Declarative\pause
\item Scales
\end{iframe}

\begin{frame}
    \frametitle{What to automate?}
    \huge
    Application\\
    Reverse Proxy / Databases\\
    Monitoring\\
    Operation System\\
    Platform
\end{frame}

\begin{frame}
    \frametitle{To manage or not?}
    \huge You have to be able to chose which part you will setup with Puppet.
\end{frame}
\section{Development}
\begin{frame}
    \frametitle{Use Puppet modules}
    \huge Use separate meta-modules for OS, Dependencies, Monitoring, Application\dots{} And include them only of needed.
\end{frame}

\begin{iframe}[Yes, no or noop]
\item Include or not each class
\item You can put a whole class in no-op
\item{\texttt{include myapp\_os}}
\item{\texttt{class \{'myapp\_os': noop => true,\}}}
\end{iframe}

\CenterSlide{Puppet Modules}

\begin{iframe}[External modules]
\item Pick the best ones
\item Puppetlabs modules
\item Upstream modules
\item Modules active on Github
\item Forge rating
\item Testing, doc
\item Modules that fit your usecase
\end{iframe}

\begin{frame}\LARGE Review the modules you plan to include. You do not want bad code in your app, why would you want it in the code that deploys your app?\end{frame}

\begin{iframe}[Contribute back]
\item Get feedback (peer review)
\item Easier to maintain in long term
\item Forces you to write tests
\item Help other people
\item Puppet is not your core business
\end{iframe}


\begin{iframe}[Your modules]
\item Everyone has write access
\item Follow code standards (puppet-lint)
\item Be future-proof
\item Separation between code and data
\end{iframe}

\begin{iframe}[Your puppet tree]
\item Your tree is next to your app code
\item Submodules of your app
\item Gets the same version number
\item Parameters matches your apps parameters
\end{iframe}

\begin{iframe}[Distribute your tree]
\item Package the whole tree in a package
\item Use package dependencies to pull puppet
\item Maybe add a helper script for the first run
\end{iframe}

\begin{iframe}
\item Your tree is next to your app code
\item Submodules of your app
\item Gets the same version number
\item Contains the right parameters
\end{iframe}


\section{Runtime}

\begin{iframe}[Puppet Agent or masterless]
\item Puppet has 2 modes
\item Pull your catalog
\item Apply it from files
\item Both have advantages
\item Depends on what you want
\end{iframe}
\begin{iframe}[Masterless mode]
\item One-time run
\item No daemon running
\item No need for a Puppet master
\item No exported resources
\item No PuppetDB
\end{iframe}
\begin{iframe}[Agent mode]
\item Run every X time (to be tuned)
\item Consitency check
\item Requires one master
\item Puppet daemon running (as root)
\item Reports sent to the master
\end{iframe}

\begin{iframe}[PuppetDB]
\item Requires a master
\item Stores facts and reports
\item Easy to query
\item Dashboards available
\item Exported resources
\end{iframe}

\begin{iframe}[Hiera]
\item Data separation
\item You classes should have a stable API
\item Your main class dispatches to other modules
\item Use functions:
\begin{itemize}
    \item create\_resource
    \item mysql\_deepmerge
\end{itemize}
\end{iframe}
\begin{iframe}[Automatic Parameter Lookup]
\item{\texttt{class::param: "foobar"}}
\item Available in Puppet 3+
\item Should be avoided (obscurification)
\item Nice to have for edge cases
\end{iframe}

\begin{iframe}[Security]
\item Puppet agent runs as root
\item The master runs as "puppet" user
\item Isolated on a separated host
\item PuppetDB/Server only listens to Loopback
\item Put a reverse proxy (even for server)
\end{iframe}

\begin{iframe}[Security - Master/Server]
\item Do not use autosign
\item You can rely on external CA
\item Isolate the service from the application
\item Query PuppetDB from your monitoring tool
\end{iframe}



\CenterSlide{Pre Existing Puppet}

\begin{iframe}[Pre Existing Puppet]
\item There might be a puppet setup
\item Work in a separated environment
\item Puppet hieradata in a subdirectory
\item Prefix your custom functions
\item Be careful with exported resources
\end{iframe}

\CenterSlide{Runtime}
\begin{iframe}[Deploy your Puppet tree]
\item Install your puppet tree package
\item Install the hiera files (versioned?)
\item One puppet apply to deploy a basic server
\item Then the first agent run to deploy PuppetDB and the rest
\item There is no puppetlabs-puppet module
\end{iframe}
\section{Conclusion}


\CenterSlide{Conclusion}
\begin{iframe}[Shipping with puppet code]
\item Everyone gets benefits
\item From devs to QA to customers
\item Need to review how you release
\item Need to review how you deploy your OS
\item Puppet code is part of your app
\end{iframe}
\begin{iframe}[Advantages]
\item Setup your app easily (internally and externally)
\item Get consistent deployments at several customer
\item Predict what will be deployed
\item Get a clear view of the infrastructure
\item Say bye bye to long procedures
\end{iframe}
\begin{iframe}[There is work]
\item Big cultural changes
    \begin{itemize}
        \item Root access?
        \item The shell script works\dots
        \item I can't do X anymore\dots
    \end{itemize}
\item Purge old artifacts on updates
\item Deal with your data
\item Keep that infra up to date
\end{iframe}

\begin{iframe}[Open the pandora box!]
\item Monitoring
\item Best practices enforcement
\item Repositories management
\item Bring your own tools
\end{iframe}

\thankyouSlide
\renewcommand{\insertLogo}{}
\contactSlide

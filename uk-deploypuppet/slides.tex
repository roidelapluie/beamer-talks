\titleSlide

\frame{%
    \frametitle{\$::user}
    \framesubtitle{Julien Pivotto}
    \begin{itemize}
        \item{Open-Source consultant at {\inuits{}inuits\small.eu}}
        \item{Puppet User since 2011}
        \item{Speaker/attendee at Puppetcamps}
        \item{Puppet core contributor}
        \item{Puppet(labs) modules contributor}
        \olditem{\textit{\ctr{@roidelapluie}} \ctr{on irc/twitter/github}}
    \end{itemize}

}

\inuitsSlide

\begin{iframe}[IT Nowadays (?)]
\item Virtualization, Containers, Cloud
\item Infrastructure as Code
\item Pay for services, not software
\end{iframe}
\begin{iframe}[Providing a service]
\item You know the environment
\item You do the maintenance
\item You do the upgrades
\end{iframe}
\begin{iframe}[Shipping a software]
\item What is the environment?
\item How was it setup?
\item Did you change someting?
\end{iframe}
\begin{iframe}[Two different worlds]
\item Managing everything vs delegating
\item Known environment vs Fixed environment
\item Rolling updates vs Updates at will
\item Monitoring vs end-users complains
\end{iframe}
%\frame[b]{%
%    \largepic{entreprise.jpg}
%    \begin{center}\Huge{}\ctr{}\end{center}
%    \sethlcolor{black}
%    \flushright\minuscule{Licensed under a Creative Commons Attribution 2.0 License\\https://www.flickr.com/photos/kwarz/13293732384/}
%}


\begin{iframe}[Shipped Softwares]
\item Self-hosted
\item Stateful
\item Complexity
\item Scalable
\item Lots of reqs and deps
\end{iframe}

\begin{iframe}[Distribution]
\item Software (Source code or binaries)
\item User guide
\item Installation guide
\item People who install the software
\end{iframe}

\begin{iframe}[Challenges of SW distribution]
\item Artifacts
\item Configuration
\item Security
\item HW requirements
\item SW requirements
\item Upgrades, Maintenance
\item Monitoring
\end{iframe}


\section{Introduction}
\frame[b]{%
    \largepic{packaging.jpg}
\begin{center}\Huge{}\ctr{Artifacts}\end{center}
\sethlcolor{black}
\flushright\minuscule{Licensed under a Creative Commons Attribution-ShareAlike 2.0 License\\https://www.flickr.com/photos/halfbisqued/2353845688}
}
\begin{frame}
    \frametitle{Distributing Software}
    \begin{itemize}
        \item {Plain FTP (+SCM)}
        \item {Tarball}
        \item {Self-extracting tarball}
        \item {curl|bash}
        \item {Containers? What's inside?}
        \item {Packages (.deb,~.rpm,\dots)}
    \end{itemize}
\end{frame}
\begin{frame}
    \frametitle{Packaging}
    \begin{itemize}
        \item {Consistency checks, file lists}
        \item {Dependencies resolving}
        \item {Repositories}
        \item {GPG-Signing}
        \item {Lots of tools}
        \item {Versioning}
        \item {Unique artifacts, reproducible build}
        \item {CfgMgmt integration (Puppet, Chef\dots)}
    \end{itemize}
\end{frame}

\begin{iframe}[Dependencies]
\item{Packaged also}
\item{In versioned repositories}
\item{Test your dependencies}
\item{Distribution, Upstream packages}
\item{Mirror and cherry pick from upstream repos}
\item{Limit the number of dependencies}
\end{iframe}


\begin{iframe}[Automation]
\item Automate all the things
\item OS, Monitoring, Application
\item Reproducible builds
\item Repositories management
\end{iframe}

\begin{iframe}[Cultural changes]
\item No more manual work
\item Use an appropriate toolchain
\item Each manual action has risks and cost
\item What did I change 3 years ago?
\end{iframe}

\begin{iframe}[Impact on documentation]
\item No more complex install guides
\item Puppet training needed
\item Valid for several products
\item Inspect the catalogs
\end{iframe}

\section{Scope}
\frame[b]{%
    \largepic{puppet.jpg}
    %\sethlcolor{black}
    \flushright\minuscule{Licensed under a Creative Commons Attribution 2.0 License\\https://www.flickr.com/photos/jimmcd/4859841581}
}

\begin{iframe}[Puppet]
\item Widely used Automation tool\pause
\item Very mature\pause
\item Client/Server mode\pause
\item Standalone mode\pause
\item Declarative\pause
\item Scales
\end{iframe}

\begin{frame}
    \frametitle{What to automate?}
    \huge
    Application\\
    Reverse Proxy / Databases\\
    Monitoring\\
    Operation System\\
    Platform
\end{frame}

\begin{frame}
    \frametitle{To manage or not?}
    \huge You have to be able to chose which part you will setup with Puppet.\\\large Customers have different culture or want to get control of parts of your application.
\end{frame}

\section{Development}
\begin{frame}
    \frametitle{Use Puppet modules}
    \huge Use separate meta-modules for OS, Dependencies, Monitoring, Application\dots{} And include them only of needed.
\end{frame}

\begin{iframe}[Yes, no or noop]
\item Include or not each class
\item You can put a whole class in no-op
\item {\texttt{include myapp\_os}}
\item {\texttt{class \{'myapp\_os': noop => true,\}}}
\item {Usecase: check that preconditions are met}
\end{iframe}

\CenterSlide{Puppet Modules}

\begin{iframe}[External modules]
\item Pick the best ones
\item Puppetlabs modules
\item Upstream modules
\item Modules active on Github
\item Forge rating
\item Testing, documentation
\item Modules that fit your usecase
\end{iframe}

\begin{frame}\LARGE Review the modules you plan to include.\pause\\You do not want \textit{bad} code in your application, \textit{why} would you want it in the code that deploys it?\end{frame}

\begin{iframe}[Contribute back]
\item Get feedback (peer review)
\item Easier to maintain in long term
\item Forces you to write tests
\item Help other people
\item Puppet is not your core business
\end{iframe}


\begin{iframe}[Your own modules]
\item Everyone has write access (Devs, Ops, \dots)
\item Follow Puppet code standards (style guide)
\item Be future-proof (follow best practices)
\item Separate code and data (hiera)
\item If not you core business: Publish it!
\end{iframe}

\begin{iframe}[Your puppet tree]
\item Your tree is next to your product code
\item Submodules of your main repository
\item Gets the same version number
\item Parameters matches your product parameters
\end{iframe}

\begin{iframe}[Distribute your tree]
\item Package the whole tree in a package
\item Use package dependencies to pull puppet
\item Maybe add a helper script for the first run
\end{iframe}

\begin{iframe}[Hiera]
\item Data separation
\item You classes should have a stable API
\item Your main class dispatches to other modules
\item Use functions:
\begin{itemize}
    \item create\_resource
    \item deepmerge
\end{itemize}
\end{iframe}

\begin{frame}
    \frametitle{Data separation}
    \huge Your code should be flexible enough: Hiera code is the only thing that should differ between your customers.
\end{frame}

\begin{iframe}[Automatic Parameter Lookup]
\item{\texttt{class::param: "foobar"}}
\item Available in Puppet 3+
\item Should be avoided (obscurification)
\item Nice to have for edge cases
\end{iframe}

\frame[b]{%
    \largepic{showtime.jpg}
    %\sethlcolor{black}
\begin{center}\Huge{}\ctr{Going live}\end{center}
    \flushright\minuscule{Licensed under a Creative Commons Attribution-ShareAlike 2.0 License\\https://www.flickr.com/photos/colink/15096002421}
}

\begin{iframe}[Puppet Agent or masterless]
\item Puppet has 2 modes
\item Agent mode: pull your catalog
\item Masterless mode: apply it from files
\item Both have advantages
\item Depends on what you want
\end{iframe}

\begin{iframe}[Masterless mode]
\item One-time run
\item No daemon running
\item No need for a Puppet master
\item No exported resources
\item No PuppetDB
\end{iframe}
\begin{iframe}[Agent mode]
\item Run every 30 minutes (to be tuned)
\item Requires one puppet master
\item Puppet Agent daemon running (as root)
\item Reports sent to the master
\item Consistency check over time
\end{iframe}

\begin{iframe}[PuppetDB]
\item Requires a master
\item Stores facts and reports
\item Easy to query
\item Dashboards available
\item Exported resources
\end{iframe}

\begin{iframe}[Deploy your Puppet tree]
\item Install your puppet tree package
\item Install the hiera files (versioned?)
\item One puppet apply to deploy a basic server
\item Then the first agent run to deploy PuppetDB and the rest
\item There is no puppetlabs-puppet module
\item Tools can help you with that (e.g kafo)
\end{iframe}


\begin{iframe}[Security]
\item Puppet agent runs as root
\item The master runs as "puppet" user
\item Isolated on a separated host
\item PuppetDB/Server only listens to Loopback
\item Put a reverse proxy (even for server)
\end{iframe}

\begin{iframe}[Security - Master/Server]
\item Do not use autosign
\item You can rely on external CA
\item Isolate the service from the application
\item Query PuppetDB from your monitoring tool
\end{iframe}


\CenterSlide{Pre Existing Puppet}

\begin{iframe}[Pre Existing Puppet]
\item There might be a puppet setup at customer
\item Work in a separated environment
\item Put your hiera data in a subdirectory
\item Prefix your custom functions
\item Be careful with exported resources
\end{iframe}

\section{Conclusion}


\CenterSlide{Conclusion}
\begin{iframe}[Shipping with puppet code]
\item Everyone gets benefits
\item From devs to QA to customers
\item Need to review how you release
\item Need to review how you deploy your OS
\item Puppet code is part of your product
\end{iframe}
\begin{iframe}[Advantages]
\item Setup your app easily (internally and externally)
\item Get consistent deployments at several customer
\item Predict what will be deployed
\item Get a clear view of the infrastructure
\item Say bye bye to long procedures
\end{iframe}
\begin{iframe}[There is work]
\item Big cultural changes
    \begin{itemize}
        \item Root access?
        \item The shell script works\dots
        \item I can't do X anymore\dots
    \end{itemize}
\item Purge old artifacts on updates
\item Deal with your data
\item Keep that infra up to date
\end{iframe}

\begin{iframe}[Open the pandora box!]
\item Monitoring
\item Best practices enforcement
\item Repositories management
\item Bring your own tools
\end{iframe}

\thankyouSlide
\renewcommand{\insertLogo}{}
\contactSlide
